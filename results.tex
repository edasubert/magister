\documentclass[text.tex]{subfiles}

\begin{document}
In this chapter we present results we have obtained using the methods described in previous chapter. For each rotational symmetry we have selected a Pisot-cyclotomic number and analyzed both one dimensional and two dimensional quasicrystals. 

For each Pisot-cyclotomic number we first show the number itself, several of its properties, then results of analysis of one dimensional quasicrystal and lastly results of analysis of two dimensional quasicrystal. 

Results for one dimensional quasicrystal entail table listing the set of distance between consecutive points of the quasicrystal for a window of certain size and two diagrams listing all possible Voronoi tiles and their periods. 

Results for two dimensional quasicrystals were acquired following the following methodology:
\begin{enumerate}
\item acquire the list of Voronoi tiles for quasicrystal with window size $\beta$
\item calculate $\omega_1$ for each Voronoi tile from the list, add each $\omega_1\in(1,\beta]$ to list \texttt{singular}
\item acquire the list of Voronoi tiles for each quasicrystal with window size on the list \texttt{singular}
\item repeat two previous steps until the list \texttt{singular} no longer grows
\item acquire the list of Voronoi tiles for each quasicrystal with window size as mean of two consecutive values on the list \texttt{singular}
\end{enumerate}

Further we present every division of window and the list of Voronoi tiles that was acquired during this process. To conserve space we do show only a $\frac{1}{16}$th or $\frac{1}{24}$th of each window since the rest is its reflection or rotation (Figure \ref{fig_results_section}) and for each Voronoi tile that appears in up to $16$ or $24$ possible orientations (rotation and reflection) we also only show one representative. 

\begin{figure}[h!]
\centering
\includegraphics[width=0.3\textwidth]{img/results/section}
\caption{The $\frac{1}{16}$th of a division of window that we present in our results. }
\label{fig_results_section}
\end{figure}

{\huge další obrázek s orientací dlaždiček}
\end{document}
