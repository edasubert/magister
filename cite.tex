\documentclass[text.tex]{subfiles}

\begin{document}
\begin{thebibliography}{9}

\bibitem{nobel}
\emph{The Nobel Prize in Chemistry 2011 - Popular Information}. Nobelprize.org. Nobel Media AB 2014. Web. 3 May 2017. \url{http://www.nobelprize.org/nobel_prizes/chemistry/laureates/2011/popular.html}

\bibitem{lagarias}
J~C~LAGARIAS. \emph{Geometric Models for Quasicrystals I. Delone Sets of Finite Type.} Discrete \& Computational Geometry 21.2 (1999): 161-91. Web.

\bibitem{number}
Z~MASÁKOVÁ, E~PELANTOVÁ. \emph{Teorie čísel}. V Praze: České vysoké učení technické v Praze, 2017. 178 stran. ISBN 978-80-01-06030-8.

\bibitem{distances}
Z~MASÁKOVÁ, J~PATERA, E PELANTOVÁ. \emph{Lattice-like properties of quasicrystal models with quadratic irrationalities},  Proceedings of Quantum Theory and Symmetries, Goslar, 1999, Eds. H.D. Doebner, V.K. Dobrev, J.D. Hennig, W. Luecke, World Scientific, 2000, pp. 499-509.

\bibitem{classification}
	Z~MASÁKOVÁ, J PATERA, J ZICH. \emph{Classification of Voronoi and Delone tiles in quasicrystals: I. General method.} J. Phys. A \textbf{36} (2003), 1869--1894.

\bibitem{classificationII}
	Z~MASÁKOVÁ, J PATERA, J ZICH. \emph{Classification of Voronoi and Delone tiles of quasicrystals: II. Circular acceptance window of arbitrary size.} J. Phys. A \textbf{36} (2003), 1895--1912.

\bibitem{classificationIII}
	Z~MASÁKOVÁ, J PATERA, J ZICH. \emph{Classification of Voronoi and Delone tiles of quasicrystals: III. Decagonal acceptance window of any size.} J. Phys. A \textbf{38} (2005), 1947--1960.

\bibitem{combinatorial}
	L-S GUIMOND, Z~MASÁKOVÁ, E PELANTOVÁ. \emph{Combinatorial properties of infinite words associated with cut-and-project
 sequences.} J. Théor. Nombres Bordeaux \textbf{15} (2003), 697--725.
	
\bibitem{magister}
	J ZICH. \emph{Voronoi \& Delone tiling of quasicrystals}. Fakulta jaderná a fyzikálně inženýrská České vysoké učení technické v~Praze, 2002. Diplomová práce. Vedoucí práce Z~Masáková.
	
\end{thebibliography}
\end{document}
