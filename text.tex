\documentclass[a4paper,10pt,twoside]{report}

\usepackage[utf8]{inputenc}
\usepackage[IL2]{fontenc}

\usepackage[english]{babel}

\PassOptionsToPackage{table}{xcolor}

\usepackage{amsmath}
\usepackage{amsthm}
\usepackage{amssymb}
\usepackage{thmtools}
\usepackage{mathtools}
\usepackage{amsfonts}
%\usepackage{fullpage}
%\usepackage{showframe}
\usepackage{booktabs}
\usepackage{multirow}
\usepackage{dcolumn}
\usepackage{tikz}
\usetikzlibrary{shapes.geometric, arrows, calc, intersections, decorations.pathmorphing}
\usepackage{subfiles}
\usepackage[ruled,czech]{algorithm2e}
\usepackage[figuresright]{rotating}
\usepackage[table]{xcolor}
\usepackage{caption}
\usepackage{subcaption}
\usepackage{geometry}
\usepackage{pdflscape}
\usepackage{longtable}
\usepackage{multicol}
\usepackage{alltt}



\usepackage[capbesideposition=inside,facing=yes,capbesidesep=quad]{floatrow}
\usepackage[maxfloats=25]{morefloats}

\usepackage{hyperref}



% Theorem Styles
\newtheorem{theorem}{Theorem}[chapter]
\newtheorem{lemma}[theorem]{Lemma}
\newtheorem{proposition}[theorem]{Proposition}
\newtheorem{corollary}[theorem]{Corollary}
% Definition Styles
%\theoremstyle{definition}
%\newtheorem{definition}[theorem]{Definition}
\declaretheorem[style=definition,qed=$\lrcorner$,numberlike=theorem]{definition}


\newtheorem{example}[theorem]{Example}
\theoremstyle{remark}
%\newtheorem{remark}{Remark}
\newtheorem*{remark}{Remark}


%\newcommand{\ring}{\mathbb{Z}\left[\beta\right]}
\newcommand{\ring}[1][\beta]{\mathbb{Z}\left[#1\right]}
\newcommand{\field}[1][\beta]{\mathbb{Q}(#1)}
\newcommand{\quasi}[1]{\Sigma(#1)}
\newcommand{\quasilist}[1]{\Psi(#1)}
\newcommand{\widequasi}[1]{\Sigma\left(#1\right)}
\newcommand{\CC}{\mathbb{C}}
\newcommand{\AAA}{\mathbb{A}}
\newcommand{\BB}{\mathbb{B}}
\newcommand{\RR}{\mathbb{R}}
\newcommand{\QQ}{\mathbb{Q}}
\newcommand{\ZZ}{\mathbb{Z}}
\newcommand{\NN}{\mathbb{N}}
\newcolumntype{d}[1]{ D{:}{:}{-1} }

\newcommand{\subspace}{\subset\subset}

\renewcommand{\i}{\mathrm{i}}


% table row height
\renewcommand{\arraystretch}{1.5}

% paragraph spacing
\setlength{\parskip}{0.4em}

\begin{document}

%%%%%%%%%%%%%%%%%%%%%%%%%%%%%%%%%%%%%%%%%%%%%%%%%%%%%%%%%%%%%%%%%%%%%%%%%%%%%%%%%%%%%%%%%%%%%%%%%%
%% TITULNÍ STRANA PRÁCE
\clearpage
\thispagestyle{empty}
\cleardoublepage

\thispagestyle{empty}

\begin{center}

{\Large ČESKÉ VYSOKÉ UČENÍ TECHNICKÉ V~PRAZE} \\[3.5mm]
{\Large Fakulta jaderná a fyzikálně inženýrská} \\[3.5mm]
{\Large Katedra matematiky}

\vspace{\stretch{0.75}}

{\Large DIPLOMOVÁ PRÁCE}

\vspace{\stretch{0.5}}

{\LARGE
\textbf{Voronoiova dláždění kvazikrystalů}
\par}

\vspace{1cm}

{\LARGE
\textbf{Voronoi tessellation of quasicrystals}
\par}

\vspace{\stretch{1.25}}

\end{center}

\begin{tabular}{ll} 
{\Large Vypracoval:} & {\Large Eduard Šubert} \\[1mm]
{\Large \v{S}kolitel:} & {\Large Ing. Petr Ambrož, Ph.D.} \\[2mm]
{\Large Akademický rok:}     & {\Large 2016/2017}
\end{tabular}

%%%%%%%%%%%%%%%%%%%%%%%%%%%%%%%%%%%%%%%%%%%%%%%%%%%%%%%%%%%%%%%%%%%%%%%%%%%%%%%%%%%%%%%%%%%%%%%%%%
%% ZADÁNÍ PRÁCE

\clearpage
\thispagestyle{empty}
\cleardoublepage

\thispagestyle{empty}

\noindent
{\Large
Na toto místo přijde svázat \textbf{zadání diplomové práce}!\\
V~jednom z~výtisků musí být \textbf{originál} zadání, v~ostatních kopie.\par}
\newpage
prázdná strana pro zadání
%%%%%%%%%%%%%%%%%%%%%%%%%%%%%%%%%%%%%%%%%%%%%%%%%%%%%%%%%%%%%%%%%%%%%%%%%%%%%%%%%%%%%%%%%%%%%%%%%%
%% ČESTNÉ PROHLÁŠENÍ

\clearpage
\thispagestyle{empty}
\cleardoublepage

\thispagestyle{empty}

\vspace*{\stretch{1}}

\noindent{\bf Čestné prohlášení}

\vspace{0.5cm}

Prohlašuji na tomto místě, že jsem předloženou práci vypracoval samostatně 
a že jsem uvedl veškerou použitou literaturu.

\vspace{1.5cm}

\noindent
\begin{minipage}[b]{5cm}
V~Praze dne \today
\end{minipage}
\hfill
\begin{minipage}[t]{5cm}
\begin{center}
\dotfill\\
Eduard Šubert
\end{center}
\end{minipage}

\vspace*{2cm}

%%%%%%%%%%%%%%%%%%%%%%%%%%%%%%%%%%%%%%%%%%%%%%%%%%%%%%%%%%%%%%%%%%%%%%%%%%%%%%%%%%%%%%%%%%%%%%%%%%
%% PODĚKOVÁNÍ

\clearpage
\thispagestyle{empty}
\cleardoublepage

\thispagestyle{empty}

\vspace*{\stretch{1}}

Děkuji Ing. Petrovi Ambrožovi, Ph.D. a prof. Ing. Zuzaně Masákové, Ph.D. za příkladné vedení mé diplomové práce a za podnětné návrhy. 

\vspace*{2cm}

%%%%%%%%%%%%%%%%%%%%%%%%%%%%%%%%%%%%%%%%%%%%%%%%%%%%%%%%%%%%%%%%%%%%%%%%%%%%%%%%%%%%%%%%%%%%%%%%%%
%% CZ/EN ABSTRAKTY A KLÍČOVÁ SLOVA

\clearpage
\thispagestyle{empty}
\cleardoublepage

\thispagestyle{empty}

{
\setlength{\parindent}{0pt}

\textit{Název práce:}
\textbf{Voronoiova dláždění kvazikrystalů} \\

\textit{Autor:} Eduard Šubert \\

\textit{Obor:} Matematická informatika \\

\textit{Druh práce:} Diplomová práce \\

\textit{Vedoucí práce:}  Ing. Petr Ambrož Ph.D., Katedra matematiky, FJFI ČVUT v~Praze \\

\textit{Abstrakt:} 
Představujeme novou vylepšenou metodu pro hledání všech různých Voronoiových polygonů v kvazikrystalu. Naše vylepšení snižuje nutné teoretické znalosti a v některý případech umožňuje najít i Voronoiovy polygony, které se vyskytují s nulovou hustotou. Naše metoda užívá přesných výpočtů, a tak naše výsledky nejsou přibližné, ale jsou přesné. 

Jako je obvyklé i náš model kvazikrystalu užívá schéma Cut-and-project. Nejdříve popisujeme naší metodu obecně, a pak ji aplikujeme na kvazikrystaly s osmi a dvanácti četnou rotační symetrií pro kruhová a osmiúhelníková/dvanáctiúhelníková okna. 

\vspace{1.5cm}

\textit{Title:}
\textbf{Voronoi tessellation of quasicrystals} \\

\textit{Author:} Eduard Šubert\\

\textit{Abstract:} 
A new improved method for finding all distinct Voronoi tiles in any quasicrystal is presented. Our improvement cuts down on necessary theoretical background and in certain cases allows one to find even the Voronoi tiles that appear with zero density. Our method uses precise arithmetic and so our results are not approximations, they are exact. 

As per usual our quasicrystal model is based on the Cut-and-project scheme. First we describe our method in general terms and then apply it to two dimensional quasicrystals with $8$-fold and $12$-fold rotational symmetries for circular and octagonal/dodecagonal windows. 


}

%%%%%%%%%%%%%%%%%%%%%%%%%%%%%%%%%%%%%%%%%%%%%%%%%%%%%%%%%%%%%%%%%%%%%%%%%%%%%%%%%%%%%%%%%%%%%%%%%%%%%
%%% Konec uvodnich stran

\clearpage
\thispagestyle{empty}
\cleardoublepage
\thispagestyle{empty}

\tableofcontents

%\clearpage
%%\thispagestyle{empty}
%%\cleardoublepage

\chapter*{Introduction}\addcontentsline{toc}{chapter}{Introduction}
\subfile{introduction}

\cleardoublepage
\chapter{Preliminaries}
\subfile{preliminaries}
%
\clearpage
\chapter{Quasicrystal}
\subfile{cyclotomic}

\clearpage
\chapter{Analysis}
\subfile{oneDimension}
\pagebreak
\subfile{twoDimension}

\clearpage
\chapter{Results}\label{cha_results}
\subfile{results}
\clearpage
\subfile{alpha}
\clearpage
\subfile{beta}
\subfile{octagon_concat}

\clearpage
\chapter{Computation}\label{cha_computation}
\subfile{computation}

\chapter*{Conclusions}\addcontentsline{toc}{chapter}{Conclusions}
\subfile{conclusion}

\subfile{cite}
\end{document}
