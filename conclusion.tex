\documentclass[text.tex]{subfiles}

\begin{document}
%There are loads of conclusions and observations to be made from our results. However due to time constraints we could not afford spending much time through the results properly. So here we would like to outline some areas that we would explore if we could as well as reflect on the work we did. 

%\section*{Compare quasicrystals for $n$-gonal and circular windows}
%The interesting property of a circle is its lack of edges. Thus every Voronoi tile that appears with zero density must appear only once, in other words two circles can never have a line segment as their intersection. We can observe that where there are sections of Voronoi tiles that appear with zero density (but more than once) in the division of $n$-gonal window there usually are sections of the same Voronoi tile in the division of the circular window, however this time they appear with nonzero density. 
%
%\section*{Compare quasicrystals for closed and open $n$-gonal windows}
%Although we did present results only for closed windows, open windows are of course also possible. Open windows make the analysis more difficult and thus we decided to use closed windows. Nevertheless we want to show at least a minimal example of the difference caused by excluding or including the border. 

%The resulting quasicrystal will differ only if the border of the window $\overline{\Omega}$ has nonempty intersection with the field $\field$. The example we show here is of quasicrystals with dodecagonal windows -- open and closed of size $\beta_{12}$. Note the change of Voronoi tiles that appear with zero density. Also interesting for this particular example is that there are only two Voronoi tiles that appear with nonzero density. Thus nearly the entire plane is tessellated using only these two Voronoi tiles. 

%\section*{Our contribution}
As mentioned in the introduction, this work is a successor to previous articles done by our colleges that focused on the analysis of quasicrystals with $10$-fold rotational symmetry. Our method is largely based on these articles. However since the computational complexity of analysis of quasicrystals with $12$-fold rotational symmetries is much higher, we could not use their method as is. 

The improved covering radius estimate proved itself to be the key for progress in the analysis. In fact it seems to be so significant improvement that we could simplify the steps of analysis and thus decrease the amount of theoretical background necessary. 

%\section*{Conclusion}
Although we are fairly happy with the overall size of our contribution there certainly is still a lot of work left to do. 
\end{document}
