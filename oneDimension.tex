\documentclass[text.tex]{subfiles}

\begin{document}
The first step of analysis of the two dimensional quasicrystal is to create algorithm for acquiring arbitrary finite section of the quasicrystal. That is however not so simple. Luckily there is a workaround. Using specific window shape it is possible to decompose two dimensional quasicrystal into two quasicrystals with one dimensional windows. We will explain exactly what that means later, for now let's explore two dimensional quasicrystals with one dimensional windows. 

By definition the quasicrystal with one dimensional window is a set of points whose Galois isomorphism images fit on a bounded section of a line. Since the Galois isomorphism inverse image of a line is again a line then two dimensional quasicrystal with a line segment for a window is in fact one dimensional (i.e. set of points on a line). 

We present this connection between two dimensional and one dimensional quasicrystal mainly to avoid the need to explicitly define the attributes and show the properties of one dimensional quasicrystal. This way aside from the rotational symmetry all the attributes and all the properties apply to the one dimensional quasicrystal as well. 

To summarize the motivation behind the analysis of one dimensional quasicrystal is the eventual analysis of two dimensional quasicrystal. To take full advantage of previous work we view one dimensional quasicrystal as a special case of two dimensional quasicrystal, that is two dimensional quasicrystal with a line segment for a window. 

\section{One dimensional quasicrystal}
First we need to show exactly what our model of one dimensional quasicrystal is. As stated before that means specifying the sets and projection for cut-and-project scheme. 

Let $\beta$ be a quadratic Pisot-cyclotomic number of order $n$, associated to $\rho = 2\cos\left(2\pi/n\right)$. 

Further let $M = \ring$ extension ring of $\beta$ and $N = \ring[\beta']$ extension ring of $\beta$'s conjugate root.

The projection $\ast:M\rightarrow N$ is the Galois isomorphism $\sigma_1$ (often denoted as $'$). 

Lastly let $\Omega\subset N$ be bounded with nonempty interior. 

Then model of one dimensional quasicrystal linked to irrationality $\beta$ and window $\Omega$ is the set:
$$\quasi{\Omega} = \{ x \in M\; |\; x^\ast\in \Omega\} = \big\{ x \in \ring\; |\; x'\in \Omega\big\}$$

Convex bounded one dimensional window is a line segment, which is in one dimension represented by an interval, specifically we will use left-closed right-open interval $\Omega = \big[-\frac{\ell}{2}, \frac{\ell}{2}\big)$ where $\ell\in (1/\beta,1]$. 

As we see in the following breakdown, quasicrystals with different openness or closeness differ only by at most a single point. 

\begin{align*}
\quasi{(c,d)} &= \left\{ 
	\begin{array}{l l}
	\quasi{[c,d)} & c \notin \ring\\
	\quasi{[c,d)}\setminus\{c'\} & c \in \ring
	\end{array} \right. 
  \\
  \quasi{[c,d]} &= \left\{ 
	\begin{array}{l l}
	\quasi{[c,d)} & d \notin \ring\\
	\quasi{[c,d)}\cup\{d'\} & d \in \ring
	\end{array} \right.
  \\
\quasi{(c,d]} &= \left\{ 
	\begin{array}{l l}
	\quasi{(c,d)} & d \notin \ring\\
	\quasi{(c,d)}\cup\{d'\} & d \in \ring
	\end{array} \right.\\
\end{align*}

Unfortunately the addition or the removal of the single point causes occurrence of local configurations that appear only once in the entire quasicrystal. That would needlessly complicate our work and therefore we chose to only analyze left-closed right-open interval, which does not suffer from these zero density occurrences. 

According to our plan, first step of the analysis is generating arbitrary finite section of one dimensional quasicrystal. 

\subsection{Arbitrary finite section}
The picture \ref{fig_onedimensional} illustrates well, what we want to acquire -- the sequence of points on the $x$ axis. For this purpose we define the sequence of the quasicrystal. 

\begin{definition}
Strictly increasing sequence $(y_n^\Omega)_{n\in \mathbb{Z}}$ defined as $\left\{y_n^\Omega\,|\, n\in\ZZ \right\} = \quasi{\Omega}$ where $\quasi{\Omega}$ is one dimensional quasicrystal is called the \textbf{sequence of the quasicrystal $\quasi{\Omega}$}.
\end{definition}

\begin{figure}[h!]
\centering
\begin{tikzpicture}[scale=0.8]

\coordinate (O) at (0,0);
\coordinate (X) at (10,0);
\coordinate (Y) at (0,6);
\coordinate (marX) at (0.4, 0);
\coordinate (marY) at (0, 0.4);
\coordinate (marXm) at ($-1*(marX)$);
\coordinate (marYm) at ($-1*(marY)$);

\coordinate (x1) at (1,1);
\coordinate (x2) at (3.732,0.2679);

\coordinate (winy1) at (0,1.8);
\coordinate (winy2) at (0,5.1);

\coordinate (R) at (0.1,0.1);


\clip($(Y)+(marY)+(marXm)$) rectangle ($(X)+(marX)+(marYm)$);

% window
\path [fill,opacity=0.2,thin] ($(winy1)+(marXm)$) -- ($(winy1)+(marX)+(X)$) -- ($(winy2)+(marX)+(X)$) -- ($(winy2)+(marXm)$) -- cycle;

% grid
\foreach \x in {-5,...,10}
{
	
	\draw [-,thin] ($-10*(x1)+\x*(x2)$) -- ($10*(x1)+\x*(x2)$);
	\draw [-,thin] ($-10*(x2)+\x*(x1)$) -- ($10*(x2)+\x*(x1)$);
	
}
\foreach \x in {-5,...,10}
{
	\foreach \i in {0,...,20}
	{
		\filldraw ($-10*(x1)+\x*(x2) + \i*(x1)$) circle (0.04);
	}
}

% axis
\draw [-,ultra thick] ($(X)+(marX)$) -- ($(O) + (marXm)$);
\draw [-,ultra thick] ($(Y)+(marY)$) -- ($(O) + (marYm)$);

% nodes
\draw [-,thick,dashed] ($(O) + -1*(3.732,0) + 4*(1,0)$)  -- ($(O) + -1*(x2) + 4*(x1)$);
%\draw [-,thick,dashed] ($(O) + -1*(0,0.2679) + 4*(0,1)$) -- ($(O) + -1*(x2) + 4*(x1)$);
\path [fill] ($(O) + -1*(3.732,0) + 4*(1,0) + -1*(R)$) rectangle ($(O) + -1*(3.732,0) + 4*(1,0) + (R)$);
\draw [-,thick,dashed] ($(O) + -1*(3.732,0) + 5*(1,0)$)  -- ($(O) + -1*(x2) + 5*(x1)$);
%\draw [-,thick,dashed] ($(O) + -1*(0,0.2679) + 5*(0,1)$) -- ($(O) + -1*(x2) + 5*(x1)$);
\path [fill] ($(O) + -1*(3.732,0) + 5*(1,0) + -1*(R)$) rectangle ($(O) + -1*(3.732,0) + 5*(1,0) + (R)$);

\draw [-,thick,dashed] ($(O) + 0*(3.732,0) + 2*(1,0)$)  -- ($(O) + 0*(x2) + 2*(x1)$);
%\draw [-,thick,dashed] ($(O) + 0*(0,0.2679) + 2*(0,1)$) -- ($(O) + 0*(x2) + 2*(x1)$);
\path [fill] ($(O) + 0*(3.732,0) + 2*(1,0) + -1*(R)$) rectangle ($(O) + 0*(3.732,0) + 2*(1,0) + (R)$);
\draw [-,thick,dashed] ($(O) + 0*(3.732,0) + 3*(1,0)$)  -- ($(O) + 0*(x2) + 3*(x1)$);
%\draw [-,thick,dashed] ($(O) + 0*(0,0.2679) + 3*(0,1)$) -- ($(O) + 0*(x2) + 3*(x1)$);
\path [fill] ($(O) + 0*(3.732,0) + 3*(1,0) + -1*(R)$) rectangle ($(O) + 0*(3.732,0) + 3*(1,0) + (R)$);
\draw [-,thick,dashed] ($(O) + 0*(3.732,0) + 4*(1,0)$)  -- ($(O) + 0*(x2) + 4*(x1)$);
%\draw [-,thick,dashed] ($(O) + 0*(0,0.2679) + 4*(0,1)$) -- ($(O) + 0*(x2) + 4*(x1)$);
\path [fill] ($(O) + 0*(3.732,0) + 4*(1,0) + -1*(R)$) rectangle ($(O) + 0*(3.732,0) + 4*(1,0) + (R)$);
\draw [-,thick,dashed] ($(O) + 0*(3.732,0) + 5*(1,0)$)  -- ($(O) + 0*(x2) + 5*(x1)$);
%\draw [-,thick,dashed] ($(O) + 0*(0,0.2679) + 5*(0,1)$) -- ($(O) + 0*(x2) + 5*(x1)$);
\path [fill] ($(O) + 0*(3.732,0) + 5*(1,0) + -1*(R)$) rectangle ($(O) + 0*(3.732,0) + 5*(1,0) + (R)$);

\draw [-,thick,dashed] ($(O) + 1*(3.732,0) + 2*(1,0)$)  -- ($(O) + 1*(x2) + 2*(x1)$);
%\draw [-,thick,dashed] ($(O) + 1*(0,0.2679) + 2*(0,1)$) -- ($(O) + 1*(x2) + 2*(x1)$);
\path [fill] ($(O) + 1*(3.732,0) + 2*(1,0) + -1*(R)$) rectangle ($(O) + 1*(3.732,0) + 2*(1,0) + (R)$);
\draw [-,thick,dashed] ($(O) + 1*(3.732,0) + 3*(1,0)$)  -- ($(O) + 1*(x2) + 3*(x1)$);
%\draw [-,thick,dashed] ($(O) + 1*(0,0.2679) + 3*(0,1)$) -- ($(O) + 1*(x2) + 3*(x1)$);
\path [fill] ($(O) + 1*(3.732,0) + 3*(1,0) + -1*(R)$) rectangle ($(O) + 1*(3.732,0) + 3*(1,0) + (R)$);
\draw [-,thick,dashed] ($(O) + 1*(3.732,0) + 4*(1,0)$)  -- ($(O) + 1*(x2) + 4*(x1)$);
%\draw [-,thick,dashed] ($(O) + 1*(0,0.2679) + 4*(0,1)$) -- ($(O) + 1*(x2) + 4*(x1)$);
\path [fill] ($(O) + 1*(3.732,0) + 4*(1,0) + -1*(R)$) rectangle ($(O) + 1*(3.732,0) + 4*(1,0) + (R)$);

\draw [-,thick,dashed] ($(O) + 2*(3.732,0) + 2*(1,0)$)  -- ($(O) + 2*(x2) + 2*(x1)$);
%\draw [-,thick,dashed] ($(O) + 2*(0,0.2679) + 2*(0,1)$) -- ($(O) + 2*(x2) + 2*(x1)$);
\path [fill] ($(O) + 2*(3.732,0) + 2*(1,0) + -1*(R)$) rectangle ($(O) + 2*(3.732,0) + 2*(1,0) + (R)$);
\draw [-,thick,dashed] ($(O) + 2*(3.732,0) + 3*(1,0)$)  -- ($(O) + 2*(x2) + 3*(x1)$);
%\draw [-,thick,dashed] ($(O) + 2*(0,0.2679) + 3*(0,1)$) -- ($(O) + 2*(x2) + 3*(x1)$);
\path [fill] ($(O) + 2*(3.732,0) + 3*(1,0) + -1*(R)$) rectangle ($(O) + 2*(3.732,0) + 3*(1,0) + (R)$);

\node [above] at (X) {$x$};
\node [right] at (Y) {$y$};

\node [below right] at (winy1) {$c$};
\draw [-,thick] ($(winy1) + -1*(0.1,0)$) -- ($(winy1) + (0.1,0)$);
\node [above right] at (winy2) {$d$};
\draw [-,thick] ($(winy2) + -1*(0.1,0)$) -- ($(winy2) + (0.1,0)$);
\end{tikzpicture}
\caption{Illustration of one-dimensional quasicrystal. The grid is $M\times N$. On the $y$ axis there is a window $\Omega = [c,d)$. The squares on the $x$ axis are points of the quasicrystal $\quasi{\Omega}$. }
\label{fig_onedimensional}
\end{figure}

Now we would like to explore the set of all possible distances between two consecutive points of the sequence of the quasicrystal: 
$$\left\{ y^\Omega_{n+1}-y^\Omega_{n}\,|\, n\in\NN \right\}$$

For that we need an expression for $y^\Omega_n$. Let's start with the simplest window: $[-\frac{1}{2}, \frac{1}{2})$. The key here is the length of the window -- i.e. $1$. First we do a little algebraic exercise with the expression for quasicrystal:
\begin{align*}
\Sigma\left(\left[-\frac{1}{2}, \frac{1}{2}\right)\right) &= \left\{x\in\ring\,\left|\, x'\in \left[-\frac{1}{2}, \frac{1}{2}\right)\right.\right\}\\
  &= \left\{a+b\beta\,\left|\, a+b\beta'\in \left[-\frac{1}{2}, \frac{1}{2}\right),\,a,b\in\ZZ\right.\right\}\\
  &= \left\{a+b\beta\,\left|\, -\frac{1}{2}\leq a+b\beta'<\frac{1}{2},\,a,b\in\ZZ\right.\right\}\\
  &= \left\{a+b\beta\,\left|\, -\frac{1}{2}-b\beta'\leq a < \frac{1}{2}-b\beta',\,a,b\in\ZZ\right.\right\}\\
  &= \left\{\left.\left\lceil-\frac{1}{2} -b\beta'\right\rceil + b\beta\,\right|\, b\in\ZZ\right\}\\
\end{align*}

Thus we can express the sequence of points of the quasicrystal as:
$$y^{\left[-\frac{1}{2}, \frac{1}{2}\right)}_n = \left\lceil-\frac{1}{2} -n\beta'\right\rceil + n\beta$$

And for the set of distances between two consecutive points we have:
$$\left\{ \left.\left\lceil-\frac{1}{2} -(n+1)\beta'\right\rceil + (n+1)\beta - \left\lceil-\frac{1}{2} -n\beta'\right\rceil - n\beta \,\right|\, n\in\ZZ\right\}$$
$$\left\{ \left.\left\lceil-\frac{1}{2} -(n+1)\beta'\right\rceil - \left\lceil-\frac{1}{2} -n\beta'\right\rceil + \beta \,\right|\, n\in\ZZ\right\}$$
$$\left\{ \left.\left\lceil-\frac{1}{2} -n\beta' - \beta'\right\rceil - \left\lceil-\frac{1}{2} -n\beta'\right\rceil + \beta \,\right|\, n\in\ZZ\right\}$$
Because $\beta$ is Pisot we have $|\beta'|<1$. Therefore the difference between the ceils is either $0$ or $1$ and the set of distances between two consecutive points thus collapses to simple $\{\beta, \beta + 1\}$. 

With a little thought and with use of the scaling property of a qusicrystal we can expand this to any window of size $\beta^k$ where $k\in\ZZ$. 
$$\Sigma\left(\beta\left[-\frac{1}{2}, \frac{1}{2}\right)\right) = \beta'\Sigma\left(\left[-\frac{1}{2}, \frac{1}{2}\right)\right)$$
Thus for window $\left[-\frac{\beta^k}{2}, \frac{\beta^k}{2}\right)$ we have the set of distances between two consecutive points $\left\{\left|(\beta')^k\beta\right|, \left|(\beta')^k(\beta + 1)\right|\right\}$. 

Applying Vieta's formulas we have:
$$\left\{\left|\frac{c^k}{\beta^{k-1}}\right|, \left|\frac{c^k}{\beta^{k-1}}+\frac{c^k}{\beta^{k}}\right|\right\}$$
where $c$ is the constant coefficient of $\beta$'s minimal polynomial. 

In the scope of our interest we now know the set of distances between two consecutive points for window of size $1$ and just outside of our scope for window of size $\frac{1}{\beta}$. Now we need to expand this knowledge to the entire interval $(\frac{1}{\beta},1]$. 

For that we utilize the inclussion property. 
\end{document}

