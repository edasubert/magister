\documentclass[text.tex]{subfiles}

\begin{document}
The first step of analysis of two dimensional quasicrystal is to create algorithm for acquiring arbitrary finite section of the quasicrystal. That is however not so simple. Luckily there is a workaround. Using specific shape of a window it is possible to decompose two dimensional quasicrystal into two one dimensional quasicrystals. We will explain exactly what that means later. However that is the motivation for analysis of the one dimensional quasicrystal. 

\section{One dimensional quasicrystal}
First we need to specify our general quasicrystal model for one dimension. 

Let $\beta$ be a Pisot-cyclotomic number of order $n$, associated to $\rho = 2\cos\left(2\pi/n\right)$ (and $\zeta = e^{2\pi i/n}$). 

Further let $M = \ring$ extension ring of $\beta$ and $M^\ast = \ring[\beta']$ extension ring of its conjugate root.

The projection $\ast:M\rightarrow M^\ast$ is the Galois isomorphism $\sigma_1$ (often denoted as $'$). 

Lastly let $\Omega\subset M^\ast$ be bounded with nonempty interior. 

Then one dimensional quasicrystal linked to irrationality $\beta$ and window $\Omega$ is the set:
$$\quasi{\Omega} = \{ x \in M\; |\; x^\ast\in \Omega\} = \big\{ x \in \ring\; |\; x'\in \Omega\big\}$$

To state the obvious, one dimensional quasicrystal is a Delone set of points on a line. 

The one dimensional window shape for which we will analyze is left-closed right-open interval $\Omega = \big[-\frac{\ell}{2}, \frac{\ell}{2}\big)$ where $\ell\in (1/\beta,1]$. 



According to our plan, first step of analysis is generating arbitrary finite section of one dimensional quasicrystal. 

\subsection{Arbitrary finite section}


\end{document}

