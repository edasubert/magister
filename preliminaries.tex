\documentclass[text.tex]{subfiles}

\begin{document}

\section{Delone set}
Delone set is such a set that is both relatively dense and uniformly discrete. In order to characterize exactly what dense and discrete means, we define two parameters for any subset of $\CC^n$. 

\begin{definition}\label{def_deloneSetPacking}
Let $D\subset \CC^n$. Then $R_P\in\RR$
$$R_P = \frac{1}{2}\sup\left\{ r_1\in \RR\left| \forall z_1,z_2\in D, z_1\neq z_2:\; \lVert z_1-z_2\rVert >r_1 \right. \right\}$$
is called \textbf{packing radius} of the set $D$. 
\end{definition}

\begin{remark}
Open balls of packing radius centered at the points of the set are disjoint. 
\end{remark}

\begin{definition}\label{def_deloneSetCovering}
Let $D\subset \CC^n$. Then $R_C\in\RR$
$$R_C = \inf\left\{ r_2>0\left| \forall z\in\CC^n:\; B(z,r_2)\cap D \neq \emptyset \right. \right\}$$
is called \textbf{covering radius} of the set $D$. 
\end{definition}

\begin{remark}
Union of closed balls of covering radius centered at the points of the set is the entire space $\CC^n$. 
\end{remark}

\begin{definition}\label{def_deloneSet}
$D\subset \CC^n$ which has positive packing radius $R_P$ is \textbf{uniformly discrete}.\\
$D\subset \CC^n$ which has finite covering radius $R_C$ is \textbf{relatively dense}.\\
$D\subset \CC^n$ which has both positive packing radius $R_P$ and finite covering radius $R_C$ is a \textbf{Delone set}.
\end{definition}

\section{number theory}

\section{Cut-and-project scheme}\label{sec_cutAndProject}
We are using cut-and-project scheme to model quasicrystals. Here is a brief introduction into its workings. 

Roughly speaking cut-and-project is a way of selecting a subset from a larger set, in our case this larger set is a $\ring$-module. 

\begin{definition}
Let $\beta\in\RR$ be Pisot, $\ring$ its extension ring and $\{\mathbf{e}_1, \dots, \mathbf{e}_d\}$ be a basis of $\RR^d$ for $d\in\NN$. 
$$L = \bigoplus^d_{j=1}\ring \mathbf{e}_j$$
is $d$ dimensional \textbf{$\ring$-module}.
\end{definition} 

The cut-and-project scheme utilizes $2n$ dimensional $\ring$-module $L\subset\RR^{2n}$ and two more $n$ dimensional subspaces $V_1, V_2\subset\RR^{2n}$. 

Further we define two projections $\pi_1:\RR^{2n}\rightarrow V_1$ and $\pi_2:\RR^{2n}\rightarrow V_2$ such that $\pi_1|_L$ is injection and $\pi_2(L)$ is dense in $V_2$. 

That is where the 'project' part of cut-and-project comes from. The 'cut' part comes from a bounded subset $\Omega\subset V_2$ with nonempty interior usually referred to as \textbf{window}. 

All put together the cut-and-project scheme produces a subset $Q\subset V_1$:

$$Q = \{ \pi_1(x)\; |\; \pi_2(x)\in \Omega,\,  x\in L \}$$

Put in words the set $Q$ are $\pi_1$ projections of those points of $L$ whose $\pi_2$ projections fit in the window $\Omega$. 

The notation can be somewhat simplified by composing a bijection between $V_1$ and $V_2$: $\pi_2\circ\pi^{-1}_1$, usually denoted as $\ast$ and referred to as a \textbf{star map}. $Q$ then becomes:

$$Q = \{ x \in V_1\; |\; x^\ast\in \Omega \}$$

This is the form in which we will use the cut-and-project scheme. 
\end{document}
