\documentclass[text.tex]{subfiles}

\begin{document}
The Nobel Prize in Chemistry 2011 was awarded to Dan Shechtman "for the discovery of quasicrystals" \cite{nobel}. His discovery opened new areas for research in chemistry, physics as well as in mathematics. We aim to contribute to the growing body of work with our mathematical analysis of several kinds of qusicrystals. Our work is a spiritual successor to series of mathematical articles from 2003 and 2005 \cite{classification}, \cite{classificationII} and \cite{classificationIII} by our colleges that are often referenced by works of physicists (\cite{mention01}, \cite{mention02}, \cite{mention03}, \cite{mention04}). Therefore we have hopes that our work will also help to better understand this peculiar form of solid matter. 

In 1982 Dan Shechtman was studying an alloy of aluminum and manganese with electron microscope. In the diffraction pattern that the electron microscope produced he observed a $10$-fold rotational symmetry, that was thought to be impossible. After several years of convincing other scientists of correctness of his observations he succeed. Others remembered that they at some point in their careers observed similar diffraction patterns, but considered them to be mistakes, thus also crystals with $8$-fold and $12$-fold rotational symmetries in their diffraction patterns were re-discovered. Later a connection was made to previous work of Alan Mackay who used the famous Penrose tessellation to create a theoretical model of matter by placing atoms at vertexes of the Penrose tessellation. He concluded that a diffraction pattern of such model would also have $10$-fold rotational symmetry. Thus mathematical analysis of quasicrystals has started, a body of work we now aim to contribute to. 

The first chapter contains all the necessary background to study of quasicrystals. Even if you are familiar with these topics we do not recommend skipping the chapter since we define several terms that are not so universal. In Chapter 2 we define what we consider to be a quasicrystal and outline the steps of our general method of analysis. The third chapter is where we do the actual analysis and point to some difficulties where we needed to take extra care. Chapter 4 is the presentation of our results. 
\end{document}
