\documentclass[text.tex]{subfiles}

\begin{document}

Add definition of quasilattice. 

\section{Pysot-Cyclotomic numbers}
Each quasilattice is intrinsically linked to an algebraic number $\beta$. In this section we describe properties of such numbers as they relate to properties of associated quasilattices. We cover only what is necessary further in this text,  for more information on Pysot-Cyclotomic numbers \cite{}. 

In order to be quasi-periodic, quasilattice needs to be "not too dense and not too sparse", in other words it needs to be a Delone set. 

\begin{definition}[Delone set] 
Let $P\subset \RR^n$ and $\exists\, R>0, \exists\, r>0$:
$$\forall x,y\in P,\, x\neq y: r\leq \|x-y\|$$
$$\forall z\in\RR^n\, \exists\, x\in P: \|z-x\|\leq R$$
Then $P$ is called \textbf{delone} set.\\
\end{definition}

For the $n$-fold rotational symmetry of the quasilettice to appear in the linked number $\beta$, it's construction starts with 
$$\rho = 2\cos\left(\frac{2\pi}{n}\right) \quad \text{where} \; n\in\ZZ$$
Such a number is a solution to the equation 
$$x^m = a_{m-1}x^{m-1}+\dots+a_1x+a_0$$
\end{document}
