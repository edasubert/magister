\documentclass[text.tex]{subfiles}

\begin{document}

\section{Quasicrystal}
In two dimensions a quasicrystal can be viewed as a subset of complex numbers $\Lambda\subset\CC$ following these five properties:

\begin{enumerate}
\item rotational symmetry: $$\exists\,\zeta = e^{2\pi i/n}:\; \zeta\Lambda = \Lambda$$
\item dilation: $$\exists\,\beta\in\RR\setminus\{-1,1\}:\; \beta\Lambda\subset \Lambda$$
\item uniform discreteness: $$\exists\,r_1>0,\; \forall z_1,z_2\in\Lambda, z_1\neq z_2:\; |z_1-z_2|>r_1$$
\item relative density: $$\exists\,r_2>0,\; \forall z\in\CC:\; B(z,r_2)\cap\Lambda \neq \emptyset$$
\item finite local complexity: $$\forall\,\rho>0:\;\big|\{\Lambda\cap B(x,\rho)\;|\;\forall x\in\Lambda\}\big| < \infty$$
\end{enumerate}

\begin{remark}
Properties 3. and 4. together make quasicrystal to be a Delone set.
\end{remark}

It stems form these properties alone, that among other constants a quasicrystal is linked to a root of unity $\zeta$ and to a number $\beta\in\RR\setminus\{-1,1\}$. Of course not every pair $(\zeta, \beta)$ is associated with a quasicrystal.

In the next section we will go through which numbers are associated with a quasicrystal and where do they come from. 

\section{Pisot-cyclotomic numbers}
Pisot-cyclotomic numbers are Pisot and are algebraically related to roots of unity. We will use these numbers in place of $\beta$ from previous section. 

\begin{definition}\label{def_pisotCyclotomic}
Let $\rho = 2\cos\left(2\pi/n\right)$ for a given $n>4$, its associate extension ring $\ZZ[\rho]$ and $m$ order of $\rho$. A Pisot-cyclotomic number of degree $m$, of order $n$ associated to $\rho$ is a Pisot number $\beta \in \ZZ[\rho]$ such that
$$\ring = \ZZ[\rho]$$
\end{definition}

Nontrivial $n$th root of unity $\zeta = e^{2\pi i/n}$ is by definition a solution to equation
$$\zeta^{n-1}+\zeta^{n-2}+\dots+\zeta+1 = 0$$
further for $\rho = 2\cos\left(2\pi/n\right)$ it holds
$$\rho = \zeta + \bar{\zeta}\quad\Rightarrow\quad \zeta^2 = \rho\zeta - 1$$
Therefore for extension rings $\ZZ[\zeta]$ and $\ZZ[\rho]$ we have
$$\ZZ[\zeta] = \ZZ[\rho] + \ZZ[\rho]\zeta$$
and finally for Pisot-cyclotomic $\beta$ associated to $\rho$ we acquire
$$\ZZ[\zeta] = \ring + \ring\zeta$$
Such countable ring is of course $n$-fold rotationally invariant
$$\zeta^k\ZZ[\zeta] = \ZZ[\zeta]\qquad k\in\widehat{n-1}$$

To summarize $\beta$ is a real Pisot and it can be used to decompose $n$-fold rotationally invariant complex ring $\ZZ[\zeta]$ as $\ring + \ring\zeta$. 

We close this section with a list of quadratic $(m=2)$ Pisot-cyclotomic numbers. 

\begin{table}[h!]
\centering
\begin{tabular}{cccc}
$n$ & $\rho$ & $\beta$ & $\zeta$ \\
\hline
$5$   & $2 \cos\left(\frac{2\pi}{5}\right)$   & $\frac{1+\sqrt{5}}{2} $    & $e^{2i\pi/5}$ \\
$8$   & $2 \cos\left(\frac{2\pi}{8}\right)$   & $ 1+\sqrt{2} $      & $e^{2i\pi/8}$ \\
$12$  & $2 \cos\left(\frac{2\pi}{12}\right)$  & $ 1+\sqrt{3} $     & $e^{2i\pi/12}$ \\
$12$  & $2 \cos\left(\frac{2\pi}{12}\right)$  & $ 2+\sqrt{3} $                & $e^{2i\pi/12}$ \\
\end{tabular}
\caption{Pisot-cyclotomic numbers of degree $2$, of order $n$, associated to $\rho$.}
\end{table}

%For the $n$-fold rotational symmetry of the quasilettice to appear in the linked number $\beta$, it's construction starts with a set of complex equations
%$$z^n=1\qquad \text{for}\, z=1$$
%$$z^{n-1}+\dots+z+1 = 0\qquad \text{for}\, z\neq 1$$

%These equations can be for $x=z+\bar{z}$ directly deduced into an algebraic irreducible equation
%\begin{equation}\label{equ_pre_algebraic} 
%x^m = a_{m-1}x^{m-1}+\dots+a_1x+a_0\qquad m\leq\left[\frac{n-1}{2}\right],\, a_i\in\ZZ
%\end{equation} 

%Roots of the equation (\ref{equ_pre_algebraic}) have the form 
%$$\rho = \rho_1 = 2\cos\left(\frac{2\pi}{n}\right)$$
%$$\rho_j = 2\cos\left(\frac{2\pi}{n}n_j\right) \qquad 2\leq j\leq m, \, n_j\in\left\{2,\dots,\left[\frac{n-1}{2}\right]\right\}$$

%The number $\beta$ is then equal to a linear combination of powers of $\rho$
%$$\beta = \sum_{k=1}^{m-1}{d_k\rho^k}\qquad d_k\in\ZZ$$


\end{document}
