\documentclass[text.tex]{subfiles}

\begin{document}

\section{Quasicrystal}
In two dimensions a quasicrystal can be viewed as a subset of complex numbers $\Lambda\subset\CC$ following these five properties:

\begin{enumerate}
\item rotational symmetry: $$\exists\,\zeta = e^{\frac{2\pi i}{n}}:\; \zeta\Lambda = \Lambda$$
\item dilation: $$\exists\,\beta\in\RR\setminus\{-1,1\}:\; \beta\Lambda\subset \Lambda$$
\item uniform discreteness: $$\exists \epsilon_1>0,\; \forall z_1,z_2\in\Lambda, z_1\neq z_2:\; |z_1-z_2|>\epsilon_1$$
\item relative density: $$\exists \epsilon_2>0,\; \forall z\in\CC:\; B(z,\epsilon_2)\cap\Lambda \neq \emptyset$$
\item finite local complexity: $$\forall\,\rho>0:\;\big|\{\Lambda\cap B(x,\rho)\;|\;\forall x\in\Lambda\}\big| < \infty$$
\end{enumerate}

\begin{remark}
Properties 3. and 4. together make quasicrystal to be a Delone set.
\end{remark}

It stems form these properties alone, that among other constants a quasicrystal is linked to a root of unity $\zeta$ and to a number $\beta\in\RR\setminus\{-1,1\}$. Of course not every pair $(\zeta, \beta)$ is associated with a quasicrystal.

In the next section we will go through which numbers are associated with a quasicrystal and where do they come from. 

\section{Pisot-cyclotomic numbers}
Pisot-cyclotomic numbers are Pisot and are algebraically related to roots of unity. We will use these numbers in place of $\beta$ from previous section. 

\begin{definition}
Let $\rho = 2\cos\left(2\pi/n\right)$ for a given $n>4$ and its associate extension ring $\ZZ[\rho]$. A Pisot-cyclotomic number of degree $m$, of order $n$ associated to $\rho$ is a Pisot number $\beta \in \ZZ[\rho]$ such that
$$\ring = \ZZ[\rho]$$
\end{definition}

%For the $n$-fold rotational symmetry of the quasilettice to appear in the linked number $\beta$, it's construction starts with a set of complex equations
%$$z^n=1\qquad \text{for}\, z=1$$
%$$z^{n-1}+\dots+z+1 = 0\qquad \text{for}\, z\neq 1$$

%These equations can be for $x=z+\bar{z}$ directly deduced into an algebraic irreducible equation
%\begin{equation}\label{equ_pre_algebraic} 
%x^m = a_{m-1}x^{m-1}+\dots+a_1x+a_0\qquad m\leq\left[\frac{n-1}{2}\right],\, a_i\in\ZZ
%\end{equation} 

%Roots of the equation (\ref{equ_pre_algebraic}) have the form 
%$$\rho = \rho_1 = 2\cos\left(\frac{2\pi}{n}\right)$$
%$$\rho_j = 2\cos\left(\frac{2\pi}{n}n_j\right) \qquad 2\leq j\leq m, \, n_j\in\left\{2,\dots,\left[\frac{n-1}{2}\right]\right\}$$

%The number $\beta$ is then equal to a linear combination of powers of $\rho$
%$$\beta = \sum_{k=1}^{m-1}{d_k\rho^k}\qquad d_k\in\ZZ$$


\end{document}
